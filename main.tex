% our rough planning is in rough.tex - see the sidebar

\documentclass{article}
\usepackage{graphicx} % Required for inserting images
\usepackage{amssymb} % useful maths symbols package 1
\usepackage{amsmath} % useful maths symbols package 2
\usepackage{amsthm} % lets us use qed square

\usepackage[
    margin=1in,
    headheight=13.6pt
]{geometry}

%For hyperlinks in text
\usepackage{hyperref}
\usepackage{cite}

\usepackage{xcolor} % Used for colours. Needed for lstlistings arrow to be blue


\usepackage[dvipsnames]{xcolor}
\usepackage[x11names]{xcolor}

\hypersetup{
colorlinks=true,
linkcolor={Aquamarine4},
urlcolor={MidnightBlue},
citecolor={Plum},
}

\usepackage{listings} % Lets us use lstlisting environment for code. Unnecessary in this project as we ended up using algorithm environments instead
\lstset{
  basicstyle=\ttfamily,
  columns=fullflexible,
  frame=single,
  breaklines=true, % Lets lstlistings break long lines of code onto multiple lines (I think that's this bit - if not its the bits above)
  postbreak=\mbox{\textcolor{blue}{$\hookrightarrow$}\space}, % Makes lstlistings put blue arrows where it breaks code onto a new line
  keepspaces=true % Makes lstlisting keep spaces in, so that I can format indents
}

\usepackage{algorithmic} % Write algorithms
\usepackage{algorithm} % Write algorithms

\usepackage{caption} % Caption formatting tools. Lets us stick * after \caption to get rid of the Object X (e.g. Table 2, Figure 7, Algorithm 4, etc) bit at the front of a caption

\usepackage[skip=10pt plus1pt]{parskip} % Controls how we want to format new paragraphs.

\usepackage{braket} %lets me write braket notation for inner product

\usepackage{graphicx}      % Required to include images
\usepackage{subcaption}    % Required for subfigures

\usepackage{wrapfig} % Lets us wrap text around a figure


\title{Industrial maths project 2: Opinion Formation}
\author{Arden Diakhate-Palme, Charlie Duncan, Jonathan Lee, Finlay Scheib, Shuhan Xu}
\date{November 2025}

\begin{document}

\maketitle

\section*{Executive summary}


\section{Introduction}
- Modelling opinion formation and behaviour of opinions over time
- In particular interested in clustering - e.g. consensus vs polarization behaviours
- Use bounded confidence agent based model
- Extended to 2D and system with aging




\section{Basic model} %maybe think of a better title for this section

We model the evolution of opinions in a population, using an agent based models. In our model, we have $n$ agents (people), all of whom have an opinion in the range  $[0,1]$. We label the opinion of agent \(i\) at time \(t\) with \(x_{i}(t)\in[0,1]\) \cite{hegselmann2015opinion}, where \(t\) is discretised. At time $t=0$, each agent's opinion is randomly drawn from a uniform distribution on $[0,1]$, before the opinion of each person is updated at each time step by the following algorithm. The inital model uses a ``bounded confidence model" (as seen in \cite{hegselmann2015opinion} and \cite{krause2000discrete}). In such a model, agents only consider the opinions of people whose opinions lie within some ``bounded confidence interval (BCI)" \(I(i,x(t))\), consisting of those agents whose opinions lie within some bound \(R\) of their opinion (i.e. the BCI for agent \(i\) is \(I(i,x(t))=\{j:|x_{i}(t)-x_{j}(t)|\leq R\}\). At each timestep, a person \(i\) updates their opinion to simply be the average of all opinions in \(I(i,x(t))\), as described in \ref{eq:basic_model}.

\begin{equation}
    \label{eq:basic_model}
    x_{i}(t+1) = |I(i,x(t)|^{-1}\sum_{j\in I(i,x(t))}x_{j}(t)
\end{equation}

In essence, each person only ``cares" about the opinions of people who hold similar views, and moves their opinion to the average position of the people whose opinions they ``care about".


\subsection{Results for basic model}

\begin{figure}[h]
    \centering
    \begin{minipage}[b]{0.48\textwidth}
        \includegraphics[width=\textwidth]{plots/initialmodel_2.png}
        \caption{Plot of initial model with consensus formation}
        \label{fig:Initial_consensus}
    \end{minipage}
    \hfill
    \begin{minipage}[b]{0.48\textwidth}
        \includegraphics[width=\textwidth]{plots/initialmodel_1.png}
        \caption{Plot of Initial model, in which we see clustering}
        \label{fig:Initial_clustering}
    \end{minipage}
\end{figure}

We coded our model in python; the results can be seen in Figures \ref{fig:Initial_consensus} and \ref{fig:Initial_clustering}. We observed that people's opinions tend to cluster very quickly to certain values. When we have just one cluster, this is referred to as ``Consensus"; when we have multiple clusters it's called ``Polarisation" \cite{hegselmann2015opinion}. Which behaviour we is heavily dependent on the confidence bound, \(R\), which quantifies how open people are to considering opinions which differ greatly from their own.

We ran our model numerous times for various values of \(R\) to see how the number of clusters depends on \(R\). Our results can be seen in Figure \ref{fig:R_vs_clusters_experiment}. We found some sort of inverse relationship between \(R\) and average number of clusters (the exact dynamics is more complicated than can be easily computed with regression methods). For \(R\gtrsim0.49\), we always get consensus (matching the analytically expected behaviour \cite{krause2000discrete}), and for \(R\gtrsim0.30\), we get consensus a majority of the time.As \(R\) gets smaller, the number of clusters increases, asymptotically increasing to infinity as \(R\to0\) (though in practice it will cap at the number of agents in our model).


\begin{figure}
    \centering
    \begin{minipage}[b]{0.48\textwidth}
        \includegraphics[width=\textwidth]{plots/R_vs_num_of_clusters_experiment.png}
        \caption{Average number of clusters against confidence bound R, with an attempted best fit using regression on log-log data for latter R values.}
        \label{fig:R_vs_clusters_experiment}
    \end{minipage}
\end{figure}

% We experiment with $R$ in order to see its effect on group dynamics and rate of convergence. In Fig [\ref{fig:Initial_consensus}], $R=0.28$, we get what is known as `Consensus Formation', where eventually all agents converge to the same opinion. If we decrease $R$ slightly to $R=0.25$, as seen in Fig [\ref{fig:Initial_clustering}], we get what is known as `Clustering' , where people converge into two distinct pools of opinion and there is no interaction or change between the two.

% According to \cite{krause2000discrete}, if \(I(i,x(0))\cap I(j,x(0))\forall i,j\in1,\ldots,n\), then we get consensus, so should get consensus always if \(R\geq0.5\), or if initial distribution has range \(2R\).
%change maths notation later - last project comments said avoid complicated maths notations like \forall if possible


% % general model 
% \section{Idea: Two Components of the 2D Model}
% \subsection{Two Parallel 1D Opinion Models with Conditional Coupling}
% \begin{itemize}
%     \item Each agent holds two separate 1D opinions, one for each topic.
%     \item Both opinions evolve independently using the 1D bounded-confidence rule.
%     \item The two opinions are two scalar variables.
%     \item If opinion 1 becomes extreme, opinion 2 is shifted by a constant.
%     \item This represents cross-topic influence triggered only under extreme conditions.
% \end{itemize}
% \subsection{Two-Dimensional Opinion Model Based on Euclidean Distance (Appendix: Model 2)}
% \begin{itemize}
%     \item Each agent’s opinion is a 2D vector combining two topic positions.
%     \item Similarity between agents is based on the Euclidean distance in the 2D plane.
%     \item Interaction occurs only when the combined 2D distance is within radius R.
%     \item Both topics influence opinion updates simultaneously.
%     \item This model forms true 2D opinion clusters in the plane, unlike the parallel 1D approach.
% \end{itemize}



\section{Drift}
\section{Ageing}



\subsection{Modelling ageing}
Modelled with
- Add shift at each timestep (so equation becomes \(x_{i}(t+1)=\min(A_{i}+d,1)\) where \(d\) is some "drift". Note need \(\min\) to keep opinions within range \([0,1]\))
- For simplicity assumed a static populace
- Simulate dying/birth by also keeping track of ages for each person
\subsection{Results of ageing model}
- Overall behaviour?
- Affect on clusters
- Various "birth" distributions

\section{Political Shocks and Stochasticity}
It is often observed that large global events, such as the 2008 global financial crisis, can shift the political opinions of an entire population. We incorporated this idea into our model by introducing a `shock' every 10th timestep that shifted the opinion of the entire population by a randomly generated amount. However, we observed next to no effect on the behaviour of clustering and consensus formation. All this did was shift the entire population around our opinion domain and collected individuals at the extremes, which we deemed unphysical.
\par
\noindent
However, the notion of adding randomness to the system lead us to introducing this to the initial model. The initial model assumes that people tend exactly to the average of people with opinions similar to them. However, we know that this is not the case. Although, people are most certainly influenced by people of similar opinions, the size of the effect this has on people is different for each individual. To implement this we introduced a random variable drawn from a truncated normal distribution $\xi_i \sim \mathcal{N}_{trunc}(0,\sigma _s)$ for $\sigma_s$ small. Then at each time step we updated their opinion by an adjusted version of Equation (\ref{eq:basic_model}),
\begin{equation}
    x_{i}(t+1) = \xi_i + \left[|I(i,x(t)|^{-1}\sum_{j\in I(i,x(t))}x_{j}(t) \right]
\end{equation}
The results can be seen in Figure \ref{shocks}, that we still achieve characteristic behaviours of consensus formation in clusters, except all the agents in each cluster do not converge to a specific value, but a family of closely related opinions, which span almost identically a similar width to the confidence bound.
\begin{figure}[h]
\begin{center}
\includegraphics[width=0.7\textwidth]{plots/shocks.png}
\end{center}
\caption{The initial model with randomness}
\label{shocks}
\end{figure}

This is more realistic of clustering we see in real life that even with clusters of political opinions, individuals still differ from others slightly and peoples opinions may change slightly over time. We also witness some drift coming naturally from the stochasticity of the agents' opinions. We witness clusters form and then join others from these clusters drifting across the opinion domain.


\section{Opinions on multiple topics}
We next sought to expand our model to be able to handle opinions on multiple interacting topics. To do this, we began with a simple extension of our previous model to \(n\) dimensions. We represent person \(j\)'s opinions at time \(t\) with a vector \(\mathbf{x}_{j}(t)=(x_{1j}(t),\ldots,x_{nj}(t))^{T}\) where \(x_{ij}(t)\in[0,1]\) represents person \(j\)'s opinion on topic \(i\) at time \(t\). Then we extend the bounded confidence interval to multiple dimensions by calculating the "distance" between two people using a standard Euclidean norm. That is to say,
\begin{equation}
    I(i)=\{j:||\mathbf{x}_{i}-\mathbf{x}_{j}||_{2}<R\}
\end{equation}
In other words, people will take into consideration the opinion of others only if their opinions are similar across all topics.

\subsection{Results for basic multidimensional model}
The results for this model can be seen in Figures \ref{fig:2D_opinions_over_time} and \ref{fig:2D_final_opinions}. We see our opinions group into clusters as with the 1D case, but in this case each cluster has an attributed opinion on both issues. One can think of these clusters as ``ideologies". To show how this differs from simply running our model twice, we plotted both the results for this model and for naively running the basic model twice in Figure \ref{fig:2D_clusters_basic_v_Euclid}. We see that our basic model simply creates a grid of points, and as such knowing an agent's opinion on one topic tells us nothing about their opinion on another. By contrast for our better 2D model, knowing an agent's opinion on one topic allows you to pin down which cluster it's in and thus it's opinion on topic 2. 

\begin{figure}[h]
    \centering
    \includegraphics[width=1\linewidth]{plots/2D_opinions_over_time.png}
    \caption{Evolution of 2D opinions over time}
    \label{fig:2D_opinions_over_time}
    \begin{minipage}[b]{0.48\textwidth}
        \includegraphics[width=\linewidth]{plots/2D_final_opinions.png}
        \caption{Final opinion clusters for 2D opinion model}
        \label{fig:2D_final_opinions}
    \end{minipage}  
    \hfill
    \begin{minipage}[b]{0.48\textwidth}
        \includegraphics[width=\textwidth]{plots/Clusters_basic_vs_Euclidean_BCR.png}
        \caption{Comparison of basic model and non interacting case (with added stocasticity to make the dynamics more obvious)}
        \label{fig:2D_clusters_basic_v_Euclid}
    \end{minipage}
\end{figure}

\subsection{Correlation}
We next sought to model 2 correlated opinions. To do so, we implemented the following scheme:

\begin{equation}
    \begin{bmatrix}
        x_{1j}(t+1)\\
        x_{2j}(t+1)
    \end{bmatrix}
    =
    \begin{bmatrix}
        (1-w)A_{1j}(t)+wA_{2j}(t)\\
        (1-w)A_{2j}(t)+wA_{1j}(t)
    \end{bmatrix}
\end{equation}

Here \(A_{ij}(t)\) denotes the average opinion on topic \(i\) of agents in the BCR for agent \(j\) at time \(t\), and \(w\) is some weight between 0 and 0.5. This model inevitably produces perfect correlation, as seen in Figure \ref{}, essentially reproducing the 1D dynamics along the diagonal. For such simple models, we expect always to eventually get exact correlation, as there is no force opposing this. %maybe cut final sentence if pressed for space
Varying the parameter \(w\) varies how quickly we achieve perfect correlation. 





\section{Conclusions and further study}

Further study suggestions
- Mostly just forces of consensus, less forces of polarisation (just randomness and ageing). Thus add forces of polarisation
    - e.g. influencers like media/parties
    - e.g. more demographic forces (i.e. give traits that pull in certain direction representing influence of factors like home ownership, income, education, urbanisation, occupation)
- Everyone's opinions in BCR weighted the same. Should perhaps weight opinions of certain individuals more (e.g. oneself, family members, members of community, people with big media presence/loud people)
- Varying characteristics of individual agents - e.g. changing R (openness), st dev in fluctuation (volatility of views), weight on others vs own opinions (stubbornness)
- Consensus occurs very quickly - probably faster than realistic (10s of step). Include stubbornness (higher weighting on own opinion/only stepping a bit in average direction) to get more realistic timescales (and indeed allow us to pin down a timescale)
- No data verification (get data to verify validity and pin down parameters)
- Varying initial distribution








%=====================================================================================================================================================================================================
% General Model Part 
\clearpage
\appendix
\bigskip
\noindent\textbf{\Large A General Model: Theory and Interpretation \\(Shuhan Xu)}
\medskip
\section{Model 1: Initial Bounded Confidence Model}
\subsection{Model Assumptions}
Start with a simple one-dimensional model of opinion formation, which is a single-topic opinion formation process.
\begin{itemize}
    \item There are $n$ individuals in total, which indexed by $i\in\{1,\dots,n\}$.
    \item Time step $t$ is discrete,  $t=0,1,2,\dots$.
    \item Each person holds an opinion about one topic, represented by a number
    \[
    x_i(t) \in [0,1],
    \] 
    where $0$ and $1$ represent two extreme opinions on a given issue.
    \item Individuals only interact with those whose opinions are within a confidence radius $R>0$. No other influencing factors are included in the model.
    \item All updates are performed in parallel.
\end{itemize}
\subsection{Mathematical Formulation}
For each individual $i$, define the trusted set at time $t$:
\[
I_i(t) \;=\; \bigl\{\, j \in \{1,\dots,n\} \; : \; |x_i(t)-x_j(t)| \le R \,\bigr\}.
\]
The parallel update rule is the local mean:
\[
x_i(t+1) \;=\; \frac{1}{|I_i(t)|}\sum_{j \in I_i(t)} x_j(t).
\]
In vector form, letting In vector form, letting
\[
f{x}(t) =
\begin{bmatrix}
x_1(t)\\x_2(t)\\\vdots\\x_n(t)
\end{bmatrix},
\]
\[\mathbf{x}(t+1) \;=\; \mathbf{A}\bigl(\mathbf{x}(t)\bigr)\,\mathbf{x}(t),\]
where $A(\mathbf{x}(t)) = [a_{ij}(t)]$ is the averaging matrix at time $t$.
\vspace{0.5em}
\\Each entry $a_{ij}(t)$ represents how much person $i$ is influenced by person $j$. It is defined as
\[
a_{ij}(t) =
\begin{cases}
\displaystyle \frac{1}{|I_i(t)|}, & \text{if } |x_i(t) - x_j(t)| \le R, \\0, & \text{otherwise.}
\end{cases}
\]
This means that each person $i$ gives equal weight $\frac{1}{|I_i(t)|}$ to everyone
whose opinion is within a distance $R$ from their own, and ignores all others. As a result, each row of $\mathbf{A}(\mathbf{x}(t))$ sums to one, so that every person’s next opinion is the average of their trusted group.
\subsection {Interpretation}
This model shows that people tend to move their opinions closer to those who think in a similar way and ignore people with very different views.
\vspace{0.5em}
\\The parameter $R$ controls how strongly people interact with those holding different opinions: 
\begin{itemize}
    \item A smaller $R$ means each person only listens to very similar opinions, which often leads to several separate clusters.
    \item A larger $R$ means individuals consider a wider range of opinions.
\end{itemize}
\noindent In all cases, the opinions eventually stabilise into one or more clusters.
\section{Model 2: Two-Dimensional Opinion Model}
\subsection{Model Assumptions}
This model extends the one opinion to two related topics.
\begin{itemize}
    \item Each person $i$ now has a two-dimensional opinion vector
    \[
    \mathbf{x}_i(t) =
    \begin{bmatrix}
    x_{i1}(t)\\
    x_{i2}(t)
    \end{bmatrix}
    \in [0,1]^2,
    \]
    where $x_{i1}$ and $x_{i2}$ represent opinions on two different topics.
    \item Two individuals $i$ and $j$ are considered close if their opinions are similar in both topics.
    \item The interaction range is controlled by a confidence radius $R>0$.
    \item No other factors are considered, and all updates are performed in parallel.
\end{itemize}
\subsection {Mathematical Formulation}
The trusted set for each person $i$ at time $t$ is defined by
\[
I_i(t) = \bigl\{\, j : \|\mathbf{x}_i(t) - \mathbf{x}_j(t)\|_2 \le R \,\bigr\},
\]
where $\|\cdot\|_2$ denotes the Euclidean distance in $\mathbb{R}^2$.
\[
\|\mathbf{x}_i(t) - \mathbf{x}_j(t)\|_2 
= \sqrt{(x_{i1}(t) - x_{j1}(t))^2 + (x_{i2}(t) - x_{j2}(t))^2}.
\]
This formula combines the differences on both topics into a single measure 
of how far apart their opinions are. If this distance is less than or equal to $R$, 
then the two people are considered similar enough to influence each other.
\vspace{0.5em}
\\The update rule works the same way as in the one-dimensional model, but now each person has two opinions, so we take the average of two-dimensional vectors:
\[
\mathbf{x}_i(t+1) = \frac{1}{|I_i(t)|}\sum_{j \in I_i(t)} \mathbf{x}_j(t).
\]
In matrix form, let
\[\mathbf{X}(t) = 
\begin{bmatrix}
\mathbf{x}_1(t)^\top\\
\vdots\\
\mathbf{x}_n(t)^\top
\end{bmatrix}
\in \mathbb{R}^{n\times 2},
\]
then
\[
\mathbf{X}(t+1) = \mathbf{A}(\mathbf{X}(t))\,\mathbf{X}(t),
\]
where 
$\mathbf{A}(\mathbf{X}(t))=[a_{ij}(t)]$ is defined in the same way as before:
\[
a_{ij}(t) =
\begin{cases}
\dfrac{1}{|I_i(t)|}, & \text{if } \|\mathbf{x}_i(t) - \mathbf{x}_j(t)\|_2 \le R, \\[6pt]
0, & \text{otherwise.}
\end{cases}
\]
\subsection{Interpretation}
Each person updates their opinions on two topics at the same time.  
\vspace{0.5em}
\\The distance $\|\mathbf{x}_i - \mathbf{x}_j\|_2$ shows how similar two people are when both topics are considered together.
\vspace{0.5em}
\\The parameter $R$ controls how wide the interaction range is:
\begin{itemize}
    \item A smaller $R$ means people only communicate with very similar others, so several separate clusters appear in the two-dimensional space.
    \item A larger $R$ means individuals can interact with a wider group of people, which can lead to one large cluster where everyone agrees.
\end{itemize}
\noindent When the two topics are correlated, people's opinions on one topic tend to move together with their opinions on the other.
\vspace{0.5em}
\\ As a result, opinions cluster along diagonal directions in the two-dimensional space. 
\vspace{0.5em}
\\ In contrast, if the two topics are independent, people's views on one issue 
do not affect their views on the other, and the clusters appear 
more symmetric across the space.
\vspace{0.5em}
\\ In all cases, opinions will eventually stop changing and form one or more stable groups in the two-dimensional opinion space.

\section{Model 3: Age Drift Model}

In reality, people’s opinions do not remain static over their lifetime. As individuals grow older, they tend to become more stable and less likely to change their views drastically. To capture this type of change, introducing an age-dependent adjustment into the original opinion formation model.
\subsection{Model with Only Age-Dependent Drift (1D)}
Firstly, adding an age-dependent drift term $\alpha_i(a_i(t))$ to the initial bounded confidence model:
\[
x_i(t+1)
= \frac{1}{|I_i(t)|}\sum_{j\in I_i(t)}x_j(t)
+ \alpha_i(a_i(t)),
\]
The term $\alpha_i(a_i(t))$ is a piecewise constant function, which represents the {age–dependent drift of opinion for individual $i$. It describes how a person's attitude naturally shifts as they grow older, even without any social interaction.
\[
\alpha_i(t) \in \mathbb{R}.
\]
Here $a_i(t)\in[0,1]$ is the normalised age variable, where $a_i(t)=0$ denotes the youngest stage 
and $a_i(t)=1$ the oldest stage.
\[
\alpha_i\!\big(a_i(t)\big)=
\begin{cases}
\alpha^{(1)}, & 0 \le a_i(t) < a_1,\\
\alpha^{(2)}, & a_1 \le a_i(t) < a_2,\\
\alpha^{(3)}, & a_2 \le a_i(t) \le 1,
\end{cases}
\qquad 0<a_1<a_2<1.
\]
\noindent A positive $\alpha_i(a_i(t))$ means that opinions tend to move upward (such as becoming more open), while a negative value means that opinions tent to move downward (such as becoming more conservative).
\vspace{0.5em}
\\Typical settings might be:
\begin{itemize}
    \item $\alpha^{(1)}>0$ for young people,
    \item $\alpha^{(2)}\approx0$ for middle–aged people,
    \item $\alpha^{(3)}<0$ for older ones.
\end{itemize}
This formulation allows the model to reflect generational differences in opinion change. For example, younger people may have positive $\alpha_i$, and older people may have negative $\alpha_i$.
\vspace{0.5em}
\\Let
\[
\mathbf{x}(t) =
\begin{bmatrix}
x_1(t) \\
x_2(t) \\
\vdots \\
x_n(t)
\end{bmatrix}
\]
and 
\[
\mathbf{A}(\mathbf{x}(t))=[a_{ij}(t)]
\]
be the averaging matrix as defined in Model 1. Then the age–dependent drift model can be written as
\[
\mathbf{x}(t+1)
=\mathbf{A}\!\big(\mathbf{x}(t)\big)\,\mathbf{x}(t)
+ \boldsymbol{\alpha}(t).
\]
Here \[
\boldsymbol{\alpha}(t) =
\begin{bmatrix}
\alpha_1(t) \\
\alpha_2(t) \\
\vdots \\
\alpha_n(t)
\end{bmatrix}
\]
is the vector of age–dependent drift terms.
\subsubsection{Limitation of the model with only age-dependent drift}
The model with only age-dependent drift is not stable. If $\alpha_i(a_i(t))$ keeps the same sign for long enough, the opinions $x_i(t)$ will continually move in one direction and eventually exceed the range $[0,1]$. 
\vspace{0.5em}
\\ Therefore, we need to stabilise this model.
\subsection{Adding the Stability Drift (1D)}
To stabilise the model, we introduce a stability term, $\beta_i(a_i(t))>0$. This term means that people's opinions gradually tend to converge to a common point (such as the social consensus), through long-term interaction and personal thinking.
\vspace{0.5em}
\\With the stability coefficient $\beta_i(a_i(t))>0$, the 1-D model update becomes
\[
x_i(t+1) = \frac{1}{|I_i(t)|}\sum_{j\in I_i(t)} x_j(t) \;-\;\beta_i\!\big(a_i(t)\big)\,\big[\,x_i(t)-m_i\big(a_i(t),t\big)\,\big]
\;+\; \alpha_i\!\big(a_i(t)\big).
\]
Here $\alpha_i(a_i(t))$ is the age–dependent drift and $m_i(a_i(t),t)$ is the consensus drift.
\vspace{0.5em}
\\Special cases: 
\begin{itemize}
    \item As $\beta_i\equiv0$, the system is the age-denpendent drift model.
    \item As $\alpha_i\equiv0$, the system is the stability drift model.
\end{itemize}
\noindent In vector form, let
\[
\mathbf{X}(t) =
\begin{bmatrix}
x_1(t) \\
x_2(t) \\
\vdots \\
x_n(t)
\end{bmatrix}.
\]
and 
\[\mathbf{A}(\mathbf{x}(t))=[a_{ij}(t)].
\]
Then
\[
\mathbf{x}(t+1)
= \mathbf{A}\!\big(\mathbf{x}(t)\big)\,\mathbf{x}(t)
\;-\; \mathbf{B}(t)\big(\mathbf{x}(t)-\mathbf{m}(t)\big)
\;+\; \boldsymbol{\alpha}(t),
\]
where
\[
\mathbf{B}(t)=diag(\boldsymbol{\beta}(t)),
\]
\[\boldsymbol{\beta}(t) =
\begin{bmatrix}
\beta_1(a_1(t)) \\
\beta_2(a_2(t)) \\
\vdots \\
\beta_n(a_n(t))
\end{bmatrix},
\quad
\boldsymbol{\alpha}(t) =
\begin{bmatrix}
\alpha_1(a_1(t)) \\
\alpha_2(a_2(t)) \\
\vdots \\
\alpha_n(a_n(t))
\end{bmatrix},
\quad
\mathbf{m}(t) =
\begin{bmatrix}
m_1(a_1(t),t) \\
m_2(a_2(t),t) \\
\vdots \\
m_n(a_n(t),t)
\end{bmatrix}.
\]
\subsubsection{Case 1: Fixed Consensus}
If the consensus is a fixed constant, eventually all opinions converge to a single common value. 
\vspace{0.5em}
\\For example, $m_i(a_i(t),t)\equiv0.5$, which means that each person gradually converges to the neutral opinion. This represents the tendency for individuals to become moderate over time, and the model always converges to $x_i^*=0.5$ in the long run.
\subsubsection{Case 2: Dynamic Social Consensus}
If the consensus is the current population mean, this term reflecting the social conformity:
\[
m_i(a_i(t),t)=\bar{x}(t)=\frac{1}{n}\sum_{j=1}^n x_j(t),
\]
This setting describes a social conformity process, where individuals adjust their opinions to align with the group consensus.
\subsubsection{Case 3: Age–Dependent Consensus}
If the consensus depends only on age,
\[
m_i(a_i(t),t)=f(a_i(t)),
\]
where
\[
f(a_i(t)) =
\begin{cases}
f_{\text{young}}(a_i(t)), & 0 \le a_i(t) < a_1,\\
f_{\text{middle}}(a_i(t)), & a_1 \le a_i(t) < a_2,\\
f_{\text{old}}(a_i(t)), & a_2 \le a_i(t) \le 1,
\end{cases}
\qquad 0<a_1<a_2<1.
\]
\\Here, $a_i(t)\in[0,1]$ is the normalised age variable of individual $i$, where $a_i(t)=0$ represents the youngest stage and $a_i(t)=1$ the oldest. The function $f(a_i(t))$ defines the consensus of each age, indicating the point that individual $i$ tends to approach in the long run. The thresholds $a_1$ and $a_2$ divide the population into three age groups: young, middle-aged, and old. In this case, $f_{\text{young}}, f_{\text{middle}}$, and $f_{\text{old}}$ denote the preferred opinion levels for these groups. Typically, $1 \ge f_{\text{young}} > f_{\text{middle}} > f_{\text{old}} \ge 0$, meaning that younger individuals are more open, while older individuals are more conservative.
\subsubsection{Case 4: Hybrid Consensus}
Finally, if the consensus combines both social and age effects, each individual is influenced by the overall social opinion and by their own age:
\[
m_i(a_i(t),t)
= w(a_i(t))\,\bar{x}(t)
+ [1-w(a_i(t))]\,f(a_i(t)).
\]
Here $\bar{x}(t)=\dfrac{1}{n}\sum_{j=1}^{n}x_j(t)$ is the mean opinion of the population at time $t$, $f(a_i(t))$ is the age-dependent consensus defined in Case 3. And $w(a_i(t))\in[0,1]$ is a weight function shows how strongly individual $i$ follows the social average. 
\vspace{0.5em}
\\Typically,
\begin{itemize}
    \item For young people, $w\rightarrow0$, which means they follow their thinking more.
    \item For old people, $w\rightarrow1$, which means they follow social consensus more.
\end{itemize}
\vspace{0.5em}
\noindent This hybrid consensus integrates individual age, social influence and time.
\subsection{Final Model with Age Drift}
\subsubsection{One Dimension}
Combine the age-dependent drift and the stability drift:
\[
x_i(t+1)
= \frac{1}{|I_i(t)|}\sum_{j\in I_i(t)} x_j(t)
-\beta_i\!\big(a_i(t)\big)\!\left[
x_i(t)
- w_i\!\big(a_i(t)\big)\,\frac{1}{n}\sum_{j=1}^n x_j(t)
- \bigl(1-w_i\!\big(a_i(t)\big)\bigr)\,f\!\big(a_i(t)\big)
\right]
+ \alpha_i\!\big(a_i(t)\big).
\]
\begin{itemize}
    \item If $w_i(a_i(t))=1$, this system becomes to the dynamic social consensus (Case~2).
    \item If $w_i(a_i(t))=0$, this system becomes to the age-dependent consensus (Case~3).
\end{itemize}
\noindent In vector form
\[
\mathbf{x}(t+1)
= \mathbf{A}\!\big(\mathbf{x}(t)\big)\,\mathbf{x}(t)
-\mathbf{B}(t)\!\left[\mathbf{x}(t)
- \mathbf{W}(a(t))\,\mathbf{J}\,\mathbf{x}(t)
- \big(\mathbf{I}-\mathbf{W}(a(t))\big)\,\mathbf{f}\!\big(a(t)\big)
\right]
+\boldsymbol{\alpha}(t),
\]
\noindent where
\[
\begin{aligned}
\mathbf{A}\!\big(\mathbf{x}(t)\big) &= [a_{ij}(t)],\qquad
a_{ij}(t)=
\begin{cases}
\dfrac{1}{|I_i(t)|}, & |x_i(t)-x_j(t)|\le R,
\\
0, & \text{otherwise,}
\end{cases}
\\
\mathbf{x}(t) &=
\begin{bmatrix}
x_1(t)\\ \vdots\\ x_n(t)
\end{bmatrix},\qquad
\boldsymbol{\beta}(t) =
\begin{bmatrix}
\beta_1\!\big(a_1(t)\big)\\ \vdots\\ \beta_n\!\big(a_n(t)\big)
\end{bmatrix},\qquad
\boldsymbol{\alpha}(t) =
\begin{bmatrix}
\alpha_1\!\big(a_1(t)\big)\\ \vdots\\ \alpha_n\!\big(a_n(t)\big)
\end{bmatrix},\\[10pt]
\mathbf{B}(t) &= \mathrm{diag}\!\big(\boldsymbol{\beta}(t)\big) =
\begin{bmatrix}
\beta_1\!\big(a_1(t)\big) & 0 & \cdots & 0\\
0 & \beta_2\!\big(a_2(t)\big) & \cdots & 0\\
\vdots & \vdots & \ddots & \vdots\\
0 & 0 & \cdots & \beta_n\!\big(a_n(t)\big)
\end{bmatrix},\\[12pt]
\mathbf{W}\!\big(a(t)\big) &= \mathrm{diag}\!\big(w_1(a_1(t)),\dots,w_n(a_n(t))\big) =
\begin{bmatrix}
w_1\!\big(a_1(t)\big) & 0 & \cdots & 0\\
0 & w_2\!\big(a_2(t)\big) & \cdots & 0\\
\vdots & \vdots & \ddots & \vdots\\
0 & 0 & \cdots & w_n\!\big(a_n(t)\big)
\end{bmatrix},
\\
\mathbf{f}\!\big(a(t)\big) &=
\begin{bmatrix}
f\!\big(a_1(t)\big)\\ \vdots\\ f\!\big(a_n(t)\big)
\end{bmatrix},\qquad
\mathbf{1} =
\begin{bmatrix}
1\\ \vdots\\ 1
\end{bmatrix},\qquad
\mathbf{J} = \frac{1}{n}\,\mathbf{1}\mathbf{1}^{\top}.
\end{aligned}
\]
\subsubsection{Two Dimensions}
Combine the age-dependent drift and the stability drift in $\mathbb{R}^2$:
\[
\mathbf{x}_i(t+1)
= \frac{1}{|I_i(t)|}\sum_{j\in I_i(t)} \mathbf{x}_j(t)
-\beta_i\!\big(a_i(t)\big)\!\left[
\mathbf{x}_i(t)
- w_i\!\big(a_i(t)\big)\,\frac{1}{n}\sum_{j=1}^n \mathbf{x}_j(t)
- \bigl(1-w_i\!\big(a_i(t)\big)\bigr)\,\mathbf{f}\!\big(a_i(t)\big)
\right]
+ \boldsymbol{\alpha}_i\!\big(a_i(t)\big),
\]
where $\mathbf{x}_i(t),\,\mathbf{f}(a_i(t)),\,\boldsymbol{\alpha}_i(a_i(t))\in\mathbb{R}^2$.
\\
\
In vector form
\[
\mathbf{X}(t+1)
= \mathbf{A}\!\big(\mathbf{X}(t)\big)\,\mathbf{X}(t)
- \mathbf{B}(t)\!\left[
\mathbf{X}(t)
- \mathbf{W}\!\big(a(t)\big)\,\mathbf{J}\,\mathbf{X}(t)
- \big(\mathbf{I}-\mathbf{W}\!\big(a(t)\big)\big)\,\mathbf{F}\!\big(a(t)\big)
\right]
+ \boldsymbol{\alpha}(t),
\]
where
\[
\begin{aligned}
\mathbf{A}\!\big(\mathbf{X}(t)\big) &= [a_{ij}(t)],\qquad
a_{ij}(t)=
\begin{cases}
\dfrac{1}{|I_i(t)|}, & \|\mathbf{x}_i(t)-\mathbf{x}_j(t)\|_2 \le R,
\\
0, & \text{otherwise,}
\end{cases}
\\[10pt]
\mathbf{X}(t) &=
\begin{bmatrix}
\mathbf{x}_1(t)^{\top}\\[3pt]
\vdots\\[3pt]
\mathbf{x}_n(t)^{\top}
\end{bmatrix},\qquad
\boldsymbol{\beta}(t) =
\begin{bmatrix}
\beta_1\!\big(a_1(t)\big)\\[3pt] \vdots\\[3pt] \beta_n\!\big(a_n(t)\big)
\end{bmatrix},\qquad
\boldsymbol{\alpha}(t) =
\begin{bmatrix}
\boldsymbol{\alpha}_1\!\big(a_1(t)\big)^{\top}\\[3pt]
\vdots\\[3pt]
\boldsymbol{\alpha}_n\!\big(a_n(t)\big)^{\top}
\end{bmatrix},
\\[10pt]
\mathbf{B}(t) &= \mathrm{diag}\!\big(\boldsymbol{\beta}(t)\big) =
\begin{bmatrix}
\beta_1\!\big(a_1(t)\big) & 0 & \cdots & 0\\
0 & \beta_2\!\big(a_2(t)\big) & \cdots & 0\\
\vdots & \vdots & \ddots & \vdots\\
0 & 0 & \cdots & \beta_n\!\big(a_n(t)\big)
\end{bmatrix},
\\[12pt]
\mathbf{W}\!\big(a(t)\big) &= \mathrm{diag}\!\big(w_1(a_1(t)),\dots,w_n(a_n(t))\big) =
\begin{bmatrix}
w_1\!\big(a_1(t)\big) & 0 & \cdots & 0\\
0 & w_2\!\big(a_2(t)\big) & \cdots & 0\\
\vdots & \vdots & \ddots & \vdots\\
0 & 0 & \cdots & w_n\!\big(a_n(t)\big)
\end{bmatrix},
\\[10pt]
\mathbf{F}\!\big(a(t)\big) &=
\begin{bmatrix}
\mathbf{f}_1\!\big(a_1(t)\big)^{\top}\\[3pt]
\vdots\\[3pt]
\mathbf{f}_n\!\big(a_n(t)\big)^{\top}
\end{bmatrix},\qquad
\mathbf{1} =
\begin{bmatrix}
1\\ \vdots\\ 1
\end{bmatrix},\qquad
\mathbf{J} = \frac{1}{n}\,\mathbf{1}\mathbf{1}^{\top}.
\end{aligned}
\]
Here $\mathbf{A}(\mathbf{X}(t)),\,\mathbf{B}(t),\,\mathbf{W}(a(t)),\,\mathbf{J}\in\mathbb{R}^{n\times n},\,\mathbf{X}(t)\in\mathbb{R}^{n\times2}$.
\vspace{0.5em}
\\
Each individual’s two opinions evolve according to the same social and age-based weights.

\section{Model 4: Model with Rebirth}

\subsection{Model Assumptions}
In this extension, we allow individuals to ``reborn'' when they become too old. Each person still has an age variable $a_i(t)\in[0,1]$, which increases over time. When $a_i(t)$ reaches~1, the agent exits the system and is replaced by a newborn agent whose initial opinion is inherited from a parent in the current population with a small random variation. This describes how new generations inherit opinions from older ones while keeping some diversity in the population.
\begin{itemize}
    \item Each time step increases the age by a small constant $\Delta a>0$.
    \item When $a_i(t)+\Delta a>1$, the agent is replaced by a newborn with $a_i(t+1)=0$.
    \item The parent index $p(i)$ is selected uniformly at random from $\{1,\dots,n\}$.
    \item The newborn inherits the parent’s opinion with a small Gaussian perturbation.
\end{itemize}
\subsection{Mathematical Formulation}
Update all opinions using the previous Age Drift Model, for both 1-D and 2-D cases:
\[
\mathbf{X}(t+1)
= \mathbf{A}\!\big(\mathbf{X}(t)\big)\,\mathbf{X}(t)
\;-\;\mathbf{B}(t)\!\Big[
\mathbf{X}(t)
- \mathbf{W}\!\big(a(t)\big)\,\mathbf{J}\,\mathbf{X}(t)
- \big(\mathbf{I}-\mathbf{W}\!\big(a(t)\big)\big)\,\mathbf{F}\!\big(a(t)\big)
\Big]
\;+\;\boldsymbol{\alpha}(t).
\]
After updating, project all opinions back to the range [0,1], so that the value stay within valid limits:
\[
\mathbf{X}(t+1)\;\leftarrow\;\Pi_{[0,1]^d}\!\big(\mathbf{X}(t+1)\big).
\]
And consider the age update and rebirth:
\vspace{0.5em}
\\for each $i\in\{1,\dots,n\}$,
\[
a_i(t+1)=
\begin{cases}
a_i(t)+\Delta a, & a_i(t)+\Delta a \le 1,\\[4pt]
0, & a_i(t)+\Delta a > 1 \quad\text{(rebirth)}.
\end{cases}
\]
If $a_i(t+1)=0$ (rebirth), choose a parent $p(i)\sim \mathrm{Uniform}\{1,\dots,n\}$ and set the newborn opinion by
\[
\mathbf{x}_i(t+1)\;=\;\mathbf{x}_{p(i)}(t)\;+\;\boldsymbol{\varepsilon}_i,
\qquad
\boldsymbol{\varepsilon}_i \sim \mathcal{N}\!\big(\mathbf{0},\,\sigma^2 \mathbf{I}_d\big),
\]
followed by the projection $\mathbf{x}_i(t+1)\leftarrow \Pi_{[0,1]^d}\!\big(\mathbf{x}_i(t+1)\big)$.
\vspace{0.5em}
\\
Therefore,
\[
\boxed{
\mathbf{X}(t+1)
= \Pi_{[0,1]^d}\!\Big(
\mathbf{A}\!\big(\mathbf{X}(t)\big)\,\mathbf{X}(t)
- \mathbf{B}(t)\!\big[
\mathbf{X}(t)
- \mathbf{W}\!\big(a(t)\big)\,\mathbf{J}\,\mathbf{X}(t)
- \big(\mathbf{I}-\mathbf{W}\!\big(a(t)\big)\big)\,\mathbf{F}\!\big(a(t)\big)
\big]
+ \boldsymbol{\alpha}(t)
\Big)
}
\]
\[
\mathbf{X}(t) =
\begin{bmatrix}
\mathbf{x}_1(t)^{\top}\\
\vdots\\
\mathbf{x}_n(t)^{\top}
\end{bmatrix}
\in \mathbb{R}^{n\times d},
\qquad
\mathbf{x}_i(t)\in[0,1]^d.
\]
\[
I_i(t) = \{\, j:\ \|\mathbf{x}_i(t)-\mathbf{x}_j(t)\|_2 \le R \,\},
\qquad
\mathbf{A}\!\big(\mathbf{X}(t)\big) = [a_{ij}(t)],
\]
\[
a_{ij}(t)=
\begin{cases}
\dfrac{1}{|I_i(t)|}, & \text{if }\ \|\mathbf{x}_i(t)-\mathbf{x}_j(t)\|_2 \le R,\\[6pt]
0, & \text{otherwise.}
\end{cases}
\]
\[
\mathbf{B}(t)=\mathrm{diag}\!\big(\beta_1(a_1(t)),\dots,\beta_n(a_n(t))\big),
\quad
\mathbf{W}\!\big(a(t)\big)=\mathrm{diag}\!\big(w_1(a_1(t)),\dots,w_n(a_n(t))\big).
\]
\[
\mathbf{F}\!\big(a(t)\big)=
\begin{bmatrix}
\mathbf{f}_1(a_1(t))^{\top}\\
\vdots\\
\mathbf{f}_n(a_n(t))^{\top}
\end{bmatrix}\in\mathbb{R}^{n\times d},
\qquad
\boldsymbol{\alpha}(t)=
\begin{bmatrix}
\boldsymbol{\alpha}_1(a_1(t))^{\top}\\
\vdots\\
\boldsymbol{\alpha}_n(a_n(t))^{\top}
\end{bmatrix}\in\mathbb{R}^{n\times d},
\]
\[
\mathbf{f}_i(a_i(t))\in\mathbb{R}^d,\quad
\boldsymbol{\alpha}_i(a_i(t))\in\mathbb{R}^d.
\]
\[
\mathbf{1}=\begin{bmatrix}1\\ \vdots\\ 1\end{bmatrix}\in\mathbb{R}^{n},\qquad
\mathbf{J}=\frac{1}{n}\mathbf{1}\mathbf{1}^{\top}\in\mathbb{R}^{n\times n},\qquad
\mathbf{I}=\mathrm{Id}_n.
\]
\[
\Pi_{[0,1]^d}(\cdot):\ \text{projects each entry to }[0,1].
\]
\subsection{Interpretation}
This model combines all previous parts into a unified system. It describes how people’s opinions change over time under the influence of both social interaction and individual aging. At each step, individuals adjust their opinions based on those who think similarly, gradually move toward their own age-related preferences, and may experience a small drift that represents natural attitude change. The projection step keeps all opinions within the valid range $[0,1]^d$, so that no value becomes unrealistically extreme. 
\vspace{0.5em}
\\
Meanwhile, each individual’s age increases gradually. When the age reaches the upper limit, the person is replaced by a newborn with age $a_i(t+1)=0$. The newborn inherits the opinion of a randomly selected parent from the current population, with a small random difference to keep diversity. This mechanism reflects intergenerational transmission of values: new generations begin with similar ideas to the older ones but still introduce small variations. Totally, these processes prevent the society from becoming completely uniform and allow the population to maintain a dynamic balance between consensus and diversity over time.































\bibliography{bibliography}{}
\bibliographystyle{plain}

\section*{Reflective report}






\end{document}